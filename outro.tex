\indent \section{Заключение}

В процессе литературного обзора по теме работы был произведён обзор аналогов, выбран прототип и проведена его критика. Сформулированы цели и задачи моделирования модуля взаимодействия парсера с онтологиями.

В ходе моделирования был получен пакет моделей, включающий в себя следующее:
\begin{list}{\labelitemi}{\leftmargin=1.5cm}
	\item общие концептуальные модели прототипа и элементов внешней среды прототипа;
	\item базово-уровневая и модификационная модель предлагаемого решения;
	\item структурная модель предлагаемого решния;
	\item иерархическая модель информации внутри предлагаемого решения;
	\item алгоритмическая модель предлагаемого решения.
\end{list}

В ходе проектирования было выполнено техническое задание на разработку модуля взаимодействия парсера с онтологиями, уточнена структурная модель предлагаемого решения.

Выполнена инженерная реализация модуля взаимодействия парсера и онтологий, описано руководство по развёртыванию и эксплуатации демонстрационного приложения.

Данная работа в дальнейшем упростит идущие процессы централизации синтаксической информации во внешних семантических хранилищах.