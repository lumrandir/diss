\thispagestyle{empty}
\begin{center}
Министерство образования и науки  Российской Федерации\\
Федеральное государственное автономное образовательное учреждение 
высшего профессионального образования\\
<<Уральский  федеральный университет\\ имени первого Президента России Б.Н.Ельцина>>\\
Физико-технологический институт\\
Кафедра вычислительной техники\\
\end{center}

\vspace{1cm}

\begin{flushright}
		\begin{minipage}{0.50\textwidth}
			\begin{flushleft}
				ДОПУСТИТЬ К ЗАЩИТЕ В ГЭК\\
				Зав. кафедрой \hrulefill\\
				\hrulefill \space \hrulefill\\
				{\small Ф. И. О.} \hspace{3cm} {\small (подпись)}\\
				<<\rule{1.5cm}{0.2mm}>> \hrulefill \space 201\rule{1cm}{0.2mm} г.\\
			\end{flushleft}
		\end{minipage}
\end{flushright}

\vspace{1cm}

\begin{center}
{\large Развитие синтаксического парсера русского языка}\\
\vspace{1cm}
Выпускная квалификационная работа магистра\\
Пояснительная записка\\
\vspace{3cm}
	\begin{tabular}{ l c l}
		Научный руководитель, доцент, к.т.н. & \HRule & А. Г. Кудрявцев\\
		Нормоконтролер, к.т.н. & \HRule & В. В. Ковалев\\
		Студент гр. Фт-47081 & \HRule & К. С. Лукинских\\		
	\end{tabular}\\
\vspace{3cm}
Екатеринбург\\
2013
\end{center}
