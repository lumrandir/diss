\indent \section{Экономическая эффективность применения модуля взаимодействия парсера русского языка с внешними онтологиями}

\subsection{Методы определения рыночной стоимости программного продукта}

Стоимость нематериальных активов зависит от способа их приобретения. Нематериальные активы могут быть внесены в качестве вклада в уставный капитал, приобретены за плату у других организаций, получены безвозмездно или созданы на самом предприятии. Поэтому оценка может быть произведена по договоренности сторон, исходя из затрат на приобретение, по рыночной стоимости, по стоимости изготовления.

Первоначальная стоимость нематериальных активов, внесенных в счет вклада в уставный капитал организации, определяется исходя из их денежной оценки, согласованной учредителями (участниками) организации, если иное не предусмотрено законодательством РФ.

Первоначальная стоимость нематериальных активов, приобретенных за плату, определяется как сумма всех фактических расходов на приобретение (за исключением налогов на добавленную стоимость и иных возмещаемых налогов) и приведение их в состояние, пригодное для использования в запланированных целях. 

Первоначальная стоимость нематериальных активов, полученных организацией по договору дарения (безвозмездно), соответствует их рыночной стоимости на дату принятия к бухгалтерскому учету.

Первоначальная стоимость нематериальных активов, созданных самой организацией, рассчитывается как сумма всех фактических расходов на их создание, изготовление (израсходованные материальные ресурсы, оплата труда, услуги сторонних организаций, патентные пошлины, связанные с получением патентов, свидетельств, и т.п.), за исключением налогов на добавленную стоимость и иных возмещаемых налогов.

Первоначальная стоимость нематериальных активов, полученных по договорам, предусматривающим исполнение обязательств (оплату) неденежными средствами, определяется исходя из стоимости товаров, переданных или подлежащих передаче организацией. Стоимость этих товаров устанавливают исходя из цены, по которой в сравнимых обстоятельствах обычно организация определяет стоимость аналогичных товаров.

Стоимость нематериальных активов, по которой они приняты к бухгалтерскому учету, не подлежит изменению, кроме случаев, установленных законодательством РФ.

Оценка нематериальных активов, стоимость которых при приобретении определена в иностранной валюте производится в рублях путем пересчета иностранной валюты по курсу Центрального банка РФ, действующему на дату приобретения организацией объектов по праву собственности, хозяйственного ведения, оперативного управления.

В оценке нематериальных активов можно использовать три основных подхода: доходный, затратный, сравнительный.

\subsection{Сравнительный подход}

Сравнительный подход используется при оценке рыночной стоимости нематериальных активов исходя из данных о недавно совершенных сделках с аналогичными нематериальными активами. 

Сравнительный подход может применяться для тех видов нематериальных активов, сделки по которым часто совершаются на рынке. Исходной информацией для расчета стоимости объекта служат цены продажи аналогичных объектов.

Метод базируется на принципе замещения, согласно которому рациональный инвестор не заплатит за данный объект больше, чем стоимость доступного к покупке аналогичного объекта, обладающего такой же полезностью, что и данный объект. Поэтому цены продажи аналогичных объектов служат исходной информацией для расчета стоимости данного объекта.

Расчеты методами, использующими сравнительный подход осуществляются по следующим этапам.

Этап 1. Изучение соответствующего рынка и сбор информации о недавних сделках с аналогичными объектами на данном рынке. Точность расчетов в значительной мере зависит от количества и качества собранной информации. Когда информации достаточно, необходимо убедиться, что проданные объекты действительно сопоставимы с оцениваемыми нематериальными активами по своим функциям и параметрам.

Этап 2. Проверка информации. Необходимо убедиться, прежде всего в том, что цены не искажены какими-либо чрезвычайными обстоятельствами, сопутствовавшими состоявшимся сделкам. Проверяется также достоверность информации о дате сделки, физических и других характеристиках аналогичных объектов.

Этап 3. Сравнение оцениваемого объекта с каждым из аналогичных объектов и выявление отличия по дате продажи, потребительским характеристикам, местоположению, исполнению, наличию дополнительных элементов и т.д. Все различия должны быть зафиксированы и учтены.

Этап 4. Расчет стоимости данных нематериальных активов путем корректировки цен на аналогичные нематериальные активы. В той мере, в какой оцениваемый объект отличается от аналогичного, в цену последнего вносят поправки с тем, чтобы определить, по какой цене мог быть продан объект, если бы обладал теми же характеристиками, что и оцениваемый объект. При анализе цен аналогичных объектов могут применяться следующие расчетные процедуры:

\begin{list}{\labelitemi}{\leftmargin=1.5cm}
	\item определение стоимости дополнительных элементов путем парных сравнений;
	\item определение корректирующих коэффициентов, учитывающих различия между объектами по отдельным параметрам;
	\item расчет стоимости по удельным стоимостным показателям, единым для определения группы аналогичных объектов;
	\item расчет стоимости с помощью мультипликатора дохода;
	\item расчет стоимости с помощью корреляционных моделей.
\end{list}

Определение стоимости дополнительных элементов осуществляется путем сравнения цен у двух групп объектов: имеющих и не имеющих эти элементы. Например, таким образом можно определить стоимость вспомогательных устройств к станкам, вспомогательных сооружений к зданиям и т.п.

Определение корректирующих коэффициентов используется тогда, когда сравниваемые нематериальные активы различаются по отдельным техническим и размерным параметрам. Качество и уровень функционирования, комфортности, удобства обслуживания --- все эти характеристики можно учесть в стоимости введения соответствующих повышающих или понижающих коэффициентов.

Расчет стоимости по удельным показателям --- способ, применяемый в тех случаях, когда сравниваемые объекты функционально однородны, но существенно различаются по размеру и мощности. При этом выводятся удельные цены на выбранную единицу. 

Способ расчета стоимости с помощью мультипликатора дохода, представляющего собой отношения цены аналогичного объекта к ежегодному доходу его владельца; применим к тем нематериальным активам, функционирование которых приносит доход. Если оценивают нематериальные активы предприятия в целом, то применяют мультипликатор Р/Е (цена к доходу на акцию), если оценивают нематериальные активы, включающие только недвижимость предприятия, то расчет ведут с помощью мультипликатора валового рентного дохода GRM, который представляет собой отношение цены аналогичного объекта к валовой ренте его владельца. Порядок расчета такой. Для каждого аналогичного объекта рассчитывают мультипликатор дохода, затем выводят усредненное значение мультипликатора для всей группы объектов. Стоимость данного объекта получают умножением усредненного мультипликатора на прогнозируемую величину дохода от данного объекта.

Расчет стоимости нематериальных активов с помощью корреляционной модели возможен в том случае, когда имеется достаточно большое количество аналогичных объектов и можно путем статистической обработки информации построить корреляционную модель, описывающую зависимость вероятной цены объекта от 2-3 его основных параметров.

С помощью описанного подхода оценим стоимость предлагаемого решения. В качестве аналогов приведём наиболее заметные программные реализации методов, приведённых в главе 1 и прототип --- Link Grammar Parser (таблица \ref{tab:anal}). Поскольку выбранные аналоги значительно отличаются друг от друга в объёмах реализованных функций, в то время как принципиально их функции однородны, выберем метод сравнения по удельным показателям. В качестве единицы сравнения примем стоимость одного года эксплуатации (затраты на поддержку решения в случае его исходной бесплатности и т. п.).

\begin{table}[H]
\centering
\caption{Рыночная стоимость аналогов}
{\small 
\begin{tabu}to \textwidth{ | X[c] | X[c] | }
	\hline
    Аналог      & Удельная стоимость года эксплуатаци, тыс. рублей  \\ \hline
	Oracle Text (часть Oracle Database EE, реализует методы CYK и алгоритм Эрли)   & 220 \\ \hline
	ABBYY Compreno (иерархические семантики, грамматики связей) & 100 \\ \hline
	SOLARIX (контекстно-независимые грамматики, метод Эрли) & 59 \\ \hline
	DictaScope (контекстно-независимые грамматики в НФХ) & 75 \\ \hline
	Link Grammar Parser (грамматика связей)    & 120 \\ \hline
	Среднерыночное значение & 114.8 \\ \hline
\end{tabu}
}
\label{tab:anal}
\end{table} 

В таблице \ref{tab:anal2} приведены результаты сравнения рыночных стоимостей  предлагаемого решения и прототипа, рассчитанные по методу прямого анализа продаж. Из таблицы видно, что предлагаемое решение экономически более эффективно, чем прототип.

\begin{table}[H]
\centering
\caption{Сравнение рыночной стоимости прототипа и предлагаемого решения}
{\small 
\begin{tabu}to \textwidth{ | X[c] | X[c] | X[c] | }
	\hline
    Критерий      & Предлагаемое решение & Прототип  \\ \hline
    Рыночная стоимость, тыс. рублей & 114.8 & 120 \\ \hline
\end{tabu}
}
\label{tab:anal2}
\end{table} 

\subsection{Затратный подход}

На основе затратного подхода определяют стоимость воспроизводства. Хотя при затратном подходе оцененная стоимость может значительно отличаться от рыночной стоимости, так как между затратами и полезностью нет прямой связи, тем не менее встречается немало случаев, когда оправдан именно затратный подход (например, для исчисления налога на имущество, для целей страхования отдельных составляющих имущества, при судебном разделе имущества между собственниками, при распродаже имущества на открытых торгах, для бухгалтерского учета основных фондов, при переоценке основных фондов).

В условиях России, где фондовый рынок только формируется, и рыночная информация почти отсутствует, затратный подход часто оказывается единственно возможным. 

Главный признак затратного подхода --- это поэлементная оценка, то есть оцениваемые нематериальные активы расчленяются на составные части, делается оценка каждой части, а затем стоимость всех нематериальных активов получают суммированием стоимостей его частей. При этом исходят из того, что у инвестора в принципе есть возможность не только купить данные нематериальные активы, но и создать их из отдельно покупаемых элементов.  

В зависимости от характера оцениваемых нематериальных активов применяют различные методы затратного подхода. Поэтому здесь речь идет об общей последовательности расчетов по данному подходу, выполняемых в несколько этапов. 

Этап 1. Анализ структуры нематериальных активов и выделение их составных частей (компонентов), оценка стоимости которых будет производиться дифференцированно различными методами. Если нужно оценить предприятие в целом, а не только его нематериальные активы, то в нем выделяют такие компоненты как: основные фонды (земля, здания, сооружения, машины и оборудование), оборотные материальные средства, денежные средства. 

Этап 2. Выбор наиболее походящего метода оценки стоимости для каждого компонента нематериальных активов и выполнение расчётов. Для определения стоимости земельного участка применяют специальные методы, известные из теории оценки недвижимости, или расчёты ведут по ценам за 1 $м^2$, применяемым при исчислении земельного налога. 

Этап 3. Оценка реальной степени износа компонентов нематериальных активов. Термин <<износ>> в теории оценки понимается как утрата полезности объекта, а следовательно и его стоимости по различным причинам, то есть не только вследствие фактора времени. Этот термин в ином смысле употребляется в бухгалтерском учете, где под износом или амортизацией понимается механизм переноса издержек на себестоимость продукции на протяжении нормативного срока службы объекта. 

В практике оценки различают два вида износа: физический износ и моральный износ. Физический износ означает потерю физических возможностей объекта в процессе его эксплуатации. Оценка морального износа в значительно большей степени пригодна в отношении ПП. Моральный износ характеризует потерю конкурентоспособности и соответственно стоимости, в связи с появлением на рынке новых более совершенных аналогов. Моральный износ принято подразделять на технологический, функциональный и внешний.

Технологический износ является следствием влияния на стоимость достижений научно-технического прогресса в области конструкции, технологии, материалов. 

Функциональный износ есть следствие уменьшения функциональных возможностей оцениваемого объекта в сравнении с новым аналогом. 

Внешний износ проявляется в том, что объект в какой-то момент перестает отвечать новым требованиям и ограничениям, например, по экологическим причинам, безопасности и т.д. 

Основным методом определения морального износа является метод сравнения с новым, более совершенным объектом. 

Этап 4. Расчет остаточной стоимости компонентов нематериальных активов и суммарная оценка остаточной стоимости всех нематериальных активов. Остаточная стоимость на дату оценки получается вычитанием из стоимости размера накопленного износа. 

Применительно к таким объектам, как ноу-хау и изобретения, аннуитетом служат платежи роялти, то есть ежегодно выплачиваемые предприятием-лицензиатом суммы обладателю ноу-хау или патента (лицензиару), согласно заключенному между ними договору. 

Оценим стоимость предлагаемого решения на основе описанного подхода. Для учёта износа объекта оценки рассчитаем норму амортизации по формуле (4).

\begin{equation}
	K = (1 / N) \cdot 100\% 
\end{equation}

K --- норма амортизации, N --- срок полезного использования (в месяцах). Принимая срок полезного использования ПП в 5 лет, получаем норму амортизации равной 1.67\%.

Для расчёта ежемесячной суммы амортизации используется формула (5).

\begin{equation}
	A = (K / 100\%) \cdot S
\end{equation}

A --- ежемесечная сумма амортизации, K --- норма амортизации, S --- стоимость объекта, рассчитанная методом восстановленной стоимости.

Для расчёта восстановленной стоимости предлагаемого решения просуммируем затраты на его создание (таблица \ref{tab:cost}).

\begin{table}[H]
\centering
\caption{Затраты на создание предлагаемого решения}
{\small 
\begin{tabu}to \textwidth{ | X[c] | X[c] | }
	\hline
	Тип & Значение (тыс. рублей) \\ \hline
	Заработная плата разработчику в течение всего срока разработки & 90 (30 x 3) \\ \hline
    Затраты на компьютерное оборудование & 40  \\ \hline
    Затраты на электроэнергию & 9 (3 x 3) \\ \hline
    Затраты на ПО разработки & 3 \\ \hline
    Амортизация в течение срока разработки & 7.11 (1.67 * 142 * 3 / 100) \\ \hline
    Итого & 149.11 \\ \hline
\end{tabu}
}
\label{tab:cost}
\end{table} 

Аналогично рассчитаем затраты на восстановление прототипа Link Grammar Parser, учитывая срок разработки в 3 месяца и отличные от РФ заработные платы разработчикам (таблица \ref{tab:cost1}).

\begin{table}[H]
\centering
\caption{Затраты на восстановление прототипа}
{\small 
\begin{tabu}to \textwidth{ | X[c] | X[c] | }
	\hline
	Тип & Значение (тыс. рублей) \\ \hline
	Заработная плата разработчику в течение всего срока разработки & 300 (100 x 3) \\ \hline
    Затраты на компьютерное оборудование & 120  \\ \hline
    Затраты на электроэнергию & 9 (3 x 3) \\ \hline
    Затраты на ПО разработки & 0 (использовалось открытое ПО) \\ \hline
    Амортизация в течение срока разработки & 21,49 (1.67 * 429 * 3 / 100) \\ \hline
    Итого & 450.49 \\ \hline
\end{tabu}
}
\label{tab:cost1}
\end{table} 

Сравним затраты на разработку предлагаемого решения и на восстановление прототипа (таблица \ref{tab:costs}).

\begin{table}[H]
\centering
\caption{Сравнение затрат на создание предлагаемого решения и восстановление прототипа}
{\small 
\begin{tabu}to \textwidth{ | X[c] | X[c] | X[c] |}
	\hline
	Критерий & Предлагаемое решение & Прототип \\ \hline
	Стоимость разработки (тыс. рублей) & 149.11 & 450.49 \\ \hline
\end{tabu}
}
\label{tab:costs}
\end{table} 

Таким образом, по результатам оценки с использованием затратного подхода, разработка предлагаемого решения является целесообразной.

\subsection{Доходный подход}

На доходный подход опираются два наиболее распространенных метода: метод дисконтированных доходов и метод прямой капитализации. Это наиболее универсальные методы, применимые к любым видам имущественных комплексов. Метод дисконтированных доходов предполагает преобразование по определенным правилам будущих доходов, ожидаемых инвестором, в текущую стоимость оцениваемых нематериальных активов.

Бдущие доходы включают:
\begin{list}{\labelitemi}{\leftmargin=1.5cm}
	\item периодический денежный поток доходов от эксплуатации нематериальных активов на протяжении срока владения; это чистый доход инвестора, получаемый им от владения собственностью (за вычетом подоходного налога) в виде дивидендов, арендной платы и т.п.; 
	\item денежные поступления от продажи нематериальных активов в конце срока владения, то есть будущая выручка от перепродажи нематериальных активов (за вычетом издержек по оформлению сделки). 
\end{list}

Чтобы понять сущность метода дисконтированных доходов, коснемся таких понятий, как сложный процент, накопление, дисконтирование и аннуитет. 

Вложенный капитал самовозрастает по правилу сложных процентов. При этом можно указать некоторую норму (ставку) дохода, которая указывает на прирост единицы капитала по истечении определенного периода (года, квартала, месяца). В методе дисконтированных доходов норму дохода называют ставкой дисконта. 

Метод прямой капитализации достаточно прост и в этом его главное и единственное достоинство. Однако он статичен, будучи привязанным к данным одного наиболее характерного года, и поэтому требуется особое внимание к правильному выбору показателей чистого дохода и коэффициентов капитализации. Расчет текущей стоимости нематериальных активов данным методом выполняется в три последовательных этапа:
\begin{list}{\labelitemi}{\leftmargin=1.5cm}
	\item расчет ежегодного чистого дохода;
	\item выбор коэффициента капитализации; коэффициент капитализации должен быть увязан с ранее выбранным показателем капитализируемого дохода;
	\item расчет текущей стоимости нематериальных активов. 
\end{list}

Для расчёта стоимости предлагаемого решения по доходному подходу воспользуемся методом дисконтирования денежных потоков, т.к. расчёт производится на основании прогнозирования экономии на затратах.

Стоимость актива вычисляется по формуле (6).

\begin{equation}
	PV = CF_0 + \frac{1}{1 + r} \cdot CF_1 + \left( \frac{1}{1 + r} \right)^2 \cdot CF_2 + \ldots + \left( \frac{1}{1 + r} \right)^T \cdot CF_T
\end{equation}

PV --- приведённая стоимость конечной последовательности потоков $CF_0$, $CF_1$, \ldots , $CF_T$. Индекс 0 соответствует текущему году, индекс Т --- последнему году использования оцениваемого актива. r --- ставка дисконтирования.

Для оценки ставки дисконтирования воспользуемся моделью оценки долгосрочных активов. Для ее построения необходимы:
\begin{list}{\labelitemi}{\leftmargin=1.5cm}
	\item ставка доходности безрисковых активов;
	\item показатели средней доходности акций по рынку в целом;
	\item показатель риска рассматриваемой акции (фактор $\beta$).
\end{list}

Ставку доходности безрисковых активов примем на уровне ставки доходности государственных облигаций --- 7\%.

Средний показатель доходности рынка информационных технологий в РФ равен 12\%.

Фактор $\beta$ показывает риск компании в сравнении с риском для всего рынка капитала. Для расчета примем значение фактора $\beta$ равным 1.

Согласно модели оценки долгосрочных активов затраты на капитал будут вычисляться по формуле (7).
\begin{equation}
	r = r_f + \beta\cdot S_1
\end{equation}

$r_f$ --- безрисковая ставка дохода, $S_1$ --- премия за риск.

Средняя ставка роялти (R) для рынка информационных технолоий составляет 5\%. Стоимость ПП с учетом роялти расчитывается по формуле (8).
\begin{equation}
	V = PV \cdot R \cdot K_p
\end{equation}

$K_p$ --- поправочный коэффициент для стандартного роялти, равный 0.8, R --- ставка роялти.

Согласно формуле (7) ставка дисконтирования будет равна средней доходности по рынку информационных технологий, т.е. 12\%.

Результат расчёта доходности по годам для предлагаемого решения представляен в таблице \ref{tab:royal}. Поскольку предлагаемое решение распространяется бесплатно, экономия заключается в отсутствии необходимости содержания сотрудника, занимающегося его поддержкой. Учитывая специфику работ по поддержке решения (достаточно периодического удалённого вмешательства), примем затраты на такого сотрудника в 120 тыс. рублей в год.

\begin{table}[H]
\centering
\caption{Результат расчёта стоимости предлагаемого решения методом дисконтирования денежных потоков}
{\small 
\begin{tabu}to \textwidth{ | X[c] | X[c] | X[c] | X[c] | X[c] | X[c] | X[c] |}
	\hline
	& \multicolumn{6}{ c |}{Год}\\
	\hline
	& 2013 & 2014 & 2015 & 2016 & 2017 & 2018 \\
	\hline
	Экономия на затратах (тыс. рублей) & 0 & 120 & 120 & 120 & 120 & 120 \\
	\hline
	Ставка дисконтирования & 0.12 & 0.12 & 0.12 & 0.12 & 0.12 & 0.12 \\
	\hline
	$\frac{CF_T}{(1 + r)^T}$ & 0 & 107.14 & 95.66 & 85.41 & 76.26 & 68.09 \\
	\hline
	PV & \multicolumn{6}{ c |}{432.56} \\
	\hline
\end{tabu}
}
\label{tab:royal}
\end{table} 

Составим аналогичную таблицу для прототипа, учитывая, что экономию на затратах при его использовании составляет отсутствие необходимости обращаться к сторонним аналитикам и специалистам по поиску (200 тыс. рублей в год). Результат представлен в таблице \ref{tab:royal1}.

\begin{table}[H]
\centering
\caption{Результат расчёта стоимости прототипа методом дисконтирования денежных потоков}
{\small 
\begin{tabu}to \textwidth{ | X[c] | X[c] | X[c] | X[c] | X[c] | X[c] | X[c] |}
	\hline
	& \multicolumn{6}{ c |}{Год}\\
	\hline
	& 2013 & 2014 & 2015 & 2016 & 2017 & 2018 \\
	\hline
	Экономия на затратах (тыс. рублей) & 0 & 200& 200 & 200 & 200 & 200 \\
	\hline
	Ставка дисконтирования & 0.12 & 0.12 & 0.12 & 0.12 & 0.12 & 0.12 \\
	\hline
	$\frac{CF_T}{(1 + r)^T}$ & 0 & 178.57 & 159.44 & 142.36 & 127.10 & 113.49 \\
	\hline
	PV & \multicolumn{6}{ c |}{720.96} \\
	\hline
\end{tabu}
}
\label{tab:royal1}
\end{table} 

Сравним стоимости прототипа и предлагаемого решения (таблица \ref{tab:royals}).

\begin{table}[H]
\centering
\caption{Сравнение стоимости прототипа и предлагаемого решения}
{\small 
\begin{tabu}to \textwidth{ | X[c] | X[c] | X[c] |}
	\hline
	Критерий & Предлагаемое решение & Прототип \\ \hline
	Стоимость (тыс. рублей) & 432.56 & 720.96 \\ \hline
\end{tabu}
}
\label{tab:royals}
\end{table} 

Как видно из таблицы \ref{tab:royals}, использование предлагаемого решения более экономически выгодно.

\subsection{Согласование результатов}

Влияние каждого из подходов в формирование конечной стоимости определим, вычислив веса по формуле (9).
\begin{equation}
	h_i = \frac{V_i}{\sum V}
\end{equation}

Конечное согласование полученных результатов производится по формуле (10).
\begin{equation}
	V = \sum_{i = 1}^k V_i \cdot h_i
\end{equation}

Результаты согласования для предлагаемого решения и прототипа представлены в таблицах \ref{tab:balance} и \ref{tab:balance1}. Сравнение согласованных результатов приведено в таблице \ref{tab:balances}.

\begin{table}[H]
\centering
\caption{Результат расчёта взвешенной стоимости предлагаемого решения}
{\small 
\begin{tabu}to \textwidth{ | X[c] | X[c] | X[c] | X[c] | }
	\hline
	Подход & Стоимость (тыс. рублей) & Вес & Взвешенная стоимость (тыс. рублей) \\
	\hline
	Сравнительный & 114.8 & 0.16 & 18.37 \\
	\hline
	Затратный & 149.11 & 0.21 & 31.31 \\
	\hline
	Доходный & 432.56 & 0.63 & 272.51 \\
	\hline
	Итого & 696.47 & 1.00 &  322.19 \\ 
	\hline
\end{tabu}
}
\label{tab:balance}
\end{table} 

\begin{table}[H]
\centering
\caption{Результат расчёта стоимости прототипа методом дисконтирования денежных потоков}
{\small 
\begin{tabu}to \textwidth{ | X[c] | X[c] | X[c] | X[c] | }
	\hline
	Подход & Стоимость (тыс. рублей) & Вес & Взвешенная стоимость (тыс. рублей) \\
	\hline
	Сравнительный & 120 & 0.09 & 10.8 \\
	\hline
	Затратный & 450.49 & 0.35 & 157.67 \\
	\hline
	Доходный & 720.96 & 0.56 & 403.74 \\
	\hline
	Итого & 1291.45 & 1.00 & 572.21 \\
	\hline
\end{tabu}
}
\label{tab:balance1}
\end{table} 

\begin{table}[H]
\centering
\caption{Сравнение стоимости прототипа и предлагаемого решения}
{\small 
\begin{tabu}to \textwidth{ | X[c] | X[c] | X[c] |}
	\hline
	Критерий & Предлагаемое решение & Прототип \\ \hline
	Взвешенная стоимость (тыс. рублей) & 322.19 & 572.21 \\ \hline
\end{tabu}
}
\label{tab:balances}
\end{table} 

Как видно из результатов взвешенной оценки, реализация и внедрение предлагаемого решения экономически более выгодны, чем использованием прототипа.

\subsection{Результаты и выводы по главе 5}

В процессе оценки экономической целесообразности реализации предлагаемого решения были достигнуты следующие результаты:
\begin{list}{\labelitemi}{\leftmargin=1.5cm}
	\item найдена стоимость предлагаемого решения с помощью сравнительного, затратного и доходного подходов;
	\item найдена стоимость прототипа с помощью сравнительного, затратного и доходного подходов;
	\item проведено сравнение совокупных по всем подходам стоимостей прототипа и предлагаемого решения.
\end{list}

\textbf{Вывод}: стоимость предлагаемого решения ниже, чем стоимость прототипа, следовательно, разработка и внедрение предлагаемого решения экономически целесообразны.