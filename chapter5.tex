\indent \section{Экономическая эффективность применения модуля взаимодействия парсера русского языка с внешними онтологиями}

\subsection{Методы определения рыночной стоимости программного продукта}

Стоимость нематериальных активов зависит от способа их приобретения. Нематериальные активы могут быть внесены в качестве вклада в уставный капитал, приобретены за плату у других организаций, получены безвозмездно или созданы на самом предприятии. Поэтому оценка может быть произведена по договоренности сторон, исходя из затрат на приобретение, по рыночной стоимости, по стоимости изготовления.

Первоначальная стоимость нематериальных активов, внесенных в счет вклада в уставный капитал организации, определяется исходя из их денежной оценки, согласованной учредителями (участниками) организации, если иное не предусмотрено законодательством РФ.

Первоначальная стоимость нематериальных активов, приобретенных за плату, определяется как сумма всех фактических расходов на приобретение (за исключением налогов на добавленную стоимость и иных возмещаемых налогов) и приведение их в состояние, пригодное для использования в запланированных целях. 

Первоначальная стоимость нематериальных активов, полученных организацией по договору дарения (безвозмездно), соответствует их рыночной стоимости на дату принятия к бухгалтерскому учету.

Первоначальная стоимость нематериальных активов, созданных самой организацией, рассчитывается как сумма всех фактических расходов на их создание, изготовление (израсходованные материальные ресурсы, оплата труда, услуги сторонних организаций, патентные пошлины, связанные с получением патентов, свидетельств, и т.п.), за исключением налогов на добавленную стоимость и иных возмещаемых налогов.

Первоначальная стоимость нематериальных активов, полученных по договорам, предусматривающим исполнение обязательств (оплату) неденежными средствами, определяется исходя из стоимости товаров, переданных или подлежащих передаче организацией. Стоимость этих товаров устанавливают исходя из цены, по которой в сравнимых обстоятельствах обычно организация определяет стоимость аналогичных товаров.

Стоимость нематериальных активов, по которой они приняты к бухгалтерскому учету, не подлежит изменению, кроме случаев, установленных законодательством РФ.

Оценка нематериальных активов, стоимость которых при приобретении определена в иностранной валюте производится в рублях путем пересчета иностранной валюты по курсу Центрального банка РФ, действующему на дату приобретения организацией объектов по праву собственности, хозяйственного ведения, оперативного управления.

В оценке нематериальных активов можно использовать три основных подхода: доходный, затратный, сравнительный.

\subsection{Сравнительный подход}

Сравнительный подход используется при оценке рыночной стоимости нематериальных активов исходя из данных о недавно совершенных сделках с аналогичными нематериальными активами. 

Сравнительный подход может применяться для тех видов нематериальных активов, сделки по которым часто совершаются на рынке. Исходной информацией для расчета стоимости объекта служат цены продажи аналогичных объектов.

Метод базируется на принципе замещения, согласно которому рациональный инвестор не заплатит за данный объект больше, чем стоимость доступного к покупке аналогичного объекта, обладающего такой же полезностью, что и данный объект. Поэтому цены продажи аналогичных объектов служат исходной информацией для расчета стоимости данного объекта.

Расчеты методами, использующими сравнительный подход осуществляются по следующим этапам.

Этап 1. Изучение соответствующего рынка и сбор информации о недавних сделках с аналогичными объектами на данном рынке. Точность расчетов в значительной мере зависит от количества и качества собранной информации. Когда информации достаточно, необходимо убедиться, что проданные объекты действительно сопоставимы с оцениваемыми нематериальными активами по своим функциям и параметрам.

Этап 2. Проверка информации. Необходимо убедиться, прежде всего в том, что цены не искажены какими-либо чрезвычайными обстоятельствами, сопутствовавшими состоявшимся сделкам. Проверяется также достоверность информации о дате сделки, физических и других характеристиках аналогичных объектов.

Этап 3. Сравнение оцениваемого объекта с каждым из аналогичных объектов и выявление отличия по дате продажи, потребительским характеристикам, местоположению, исполнению, наличию дополнительных элементов и т.д. Все различия должны быть зафиксированы и учтены.

Этап 4. Расчет стоимости данных нематериальных активов путем корректировки цен на аналогичные нематериальные активы. В той мере, в какой оцениваемый объект отличается от аналогичного, в цену последнего вносят поправки с тем, чтобы определить, по какой цене мог быть продан объект, если бы обладал теми же характеристиками, что и оцениваемый объект. При анализе цен аналогичных объектов могут применяться следующие расчетные процедуры:

\begin{list}{\labelitemi}{\leftmargin=1.5cm}
	\item определение стоимости дополнительных элементов путем парных сравнений;
	\item определение корректирующих коэффициентов, учитывающих различия между объектами по отдельным параметрам;
	\item расчет стоимости по удельным стоимостным показателям, единым для определения группы аналогичных объектов;
	\item расчет стоимости с помощью мультипликатора дохода;
	\item расчет стоимости с помощью корреляционных моделей.
\end{list}

Определение стоимости дополнительных элементов осуществляется путем сравнения цен у двух групп объектов: имеющих и не имеющих эти элементы. Например, таким образом можно определить стоимость вспомогательных устройств к станкам, вспомогательных сооружений к зданиям и т.п.

Определение корректирующих коэффициентов используется тогда, когда сравниваемые нематериальные активы различаются по отдельным техническим и размерным параметрам. Качество и уровень функционирования, комфортности, удобства обслуживания --- все эти характеристики можно учесть в стоимости введения соответствующих повышающих или понижающих коэффициентов.

Расчет стоимости по удельным показателям --- способ, применяемый в тех случаях, когда сравниваемые объекты функционально однородны, но существенно различаются по размеру и мощности. При этом выводятся удельные цены на выбранную единицу. 

Способ расчета стоимости с помощью мультипликатора дохода, представляющего собой отношения цены аналогичного объекта к ежегодному доходу его владельца; применим к тем нематериальным активам, функционирование которых приносит доход. Если оценивают нематериальные активы предприятия в целом, то применяют мультипликатор Р/Е (цена к доходу на акцию), если оценивают нематериальные активы, включающие только недвижимость предприятия, то расчет ведут с помощью мультипликатора валового рентного дохода GRM, который представляет собой отношение цены аналогичного объекта к валовой ренте его владельца. Порядок расчета такой. Для каждого аналогичного объекта рассчитывают мультипликатор дохода, затем выводят усредненное значение мультипликатора для всей группы объектов. Стоимость данного объекта получают умножением усредненного мультипликатора на прогнозируемую величину дохода от данного объекта.

Расчет стоимости нематериальных активов с помощью корреляционной модели возможен в том случае, когда имеется достаточно большое количество аналогичных объектов и можно путем статистической обработки информации построить корреляционную модель, описывающую зависимость вероятной цены объекта от 2-3 его основных параметров.

С помощью описанного подхода оценим стоимость предлагаемого решения. В качестве аналогов приведём наиболее заметные программные реализации методов, приведённых в главе 1 и прототип --- Link Grammar Parser (таблица \ref{tab:anal}). Поскольку выбранные аналоги значительно отличаются друг от друга в объёмах реализованных функций, в то время как принципиально их функции однородны, выберем метод сравнения по удельным показателям. В качестве единицы сравнения примем стоимость одного года эксплуатации (затраты на поддержку решения в случае его исходной бесплатности и т. п.).

\begin{table}[H]
\centering
\caption{Рыночная стоимость аналогов}
{\small 
\begin{tabu}to \textwidth{ | X[c] | X[c] | }
	\hline
    Аналог      & Удельная стоимость года эксплуатаци, тыс. рублей  \\ \hline
	Oracle Text (часть Oracle Database EE, реализует методы CYK и алгоритм Эрли)   & 220 \\ \hline
	ABBYY Compreno (иерархические семантики, грамматики связей) & 100 \\ \hline
	SOLARIX (контекстно-независимые грамматики, метод Эрли) & 59 \\ \hline
	DictaScope (контекстно-независимые грамматики в НФХ) & 75 \\ \hline
	Link Grammar Parser (грамматика связей)    & 120 \\ \hline
	Среднерыночное значение & 114.8 \\ \hline
\end{tabu}
}
\label{tab:anal}
\end{table} 

В таблице \ref{tab:anal2} приведены результаты сравнения рыночных стоимостей  предлагаемого решения и прототипа, рассчитанные по методу прямого анализа продаж. Из таблицы видно, что предлагаемое решение экономически более эффективно, чем прототип.

\begin{table}[H]
\centering
\caption{Сравнение рыночной стоимости прототипа и предлагаемого решения}
{\small 
\begin{tabu}to \textwidth{ | X[c] | X[c] | X[c] | }
	\hline
    Критерий      & Предлагаемое решение & Прототип  \\ \hline
    Рыночная стоимость, тыс. рублей & 114.8 & 120 \\ \hline
\end{tabu}
}
\label{tab:anal2}
\end{table} 
