\section*{\centering Приложение А. Техническое задание на разработку модуля взаимодействия парсера русского языка с онтологиями}
\addcontentsline{toc}{section}{Приложение А. Техническое задание на разработку модуля взаимодействия парсера русского языка с онтологиями}
\newpage
\makeatletter
\def \appatableofcontents{
	\section*{\centering Содержание}\@starttoc{appatoc}
}
\makeatother
\appatableofcontents
\newpage

\section*{A1 Общие сведения}
\addcontentsline{appatoc}{section}{A1 Общие сведения}
\subsection*{A1.1 Полное наименование системы и её условное обозначение}
\addcontentsline{appatoc}{subsection}{A1.1 Полное наименование системы и её условное обозначение}
Модуль взаимодействия синтаксического парсера русского языка с внешними OWL-онтологиями. Далее Модуль.

\subsection*{A1.2 Заказчик}
\addcontentsline{appatoc}{subsection}{A1.2 Заказчик}
Федеральное государственное автономное образовательное учреждение высшего профессионального образования <<Уральский Федеральный Университет имени первого Президента России Б. Н. Ельцина>>.
\paragraph{Пользователь}: специалист по извлечению знаний, исследователь.

\subsection*{A1.3 Разработчик}
\addcontentsline{appatoc}{subsection}{A1.3 Разработчик}
Лукинских Константин Сергеевич, студент группы ФтМ-210803 <<ФГАОУ ВПО УрФУ им. первого Президента
России Б. Н. Ельцина>> Физико"=технологического института, кафедры вычислительной техники.

\subsection*{A1.4 Плановые сроки начала и окончания работ по\\ созданию системы}
\addcontentsline{appatoc}{subsection}{A1.4 Плановые сроки начала и окончания работ по созданию системы}
Начало работ: 01.03.2013.

Окончание работ: 01.06.2013.

\subsection*{A1.5 Порядок оформления и предъявления заказчику\\ результатов работ}
\addcontentsline{appatoc}{subsection}{A1.5 Порядок оформления и предъявления заказчику результатов работ}
Результаты работы предъявляются заказчику по мере окончания работ на каждом из этапов общей разработки
в соответствии с календарным планом работы.

Система в окончательном виде распространяется в виде исходного кода вместе с руководством по эксплуатации и устанавливается специалистами самостоятельно.

\section*{A2 Назначение, цели создания системы}
\addcontentsline{appatoc}{section}{A2 Назначение, цели создания системы}
\subsection*{A2.1 Назначение системы}
\addcontentsline{appatoc}{subsection}{A2.1 Назначение системы}
Система должна предоставлять интерфейс между парсером грамматик связей и внешними OWL-онтологиями, давая специалистам предметной области возможность использовать полные и централизованные источники синтаксической информации вкупе с оптимальными алгоритмами парсинга.

\subsection*{A2.2 Цели создания системы}
\addcontentsline{appatoc}{subsection}{A2.2 Цели создания системы}
\paragraph{Глобальная цель:} организовать двустороннее взаимодействие между парсером русского языка и внешней онтологией.
\paragraph{Локальные цели:}
\begin{list}{\labelitemi}{\leftmargin=1.5cm}
  \item реализовать взаимодействие с парсером посредством документированных механизмов;
  \item обеспечить доступ к онтологии с помощью сети Интернет;
  \item обеспечить двустороннее преобразование данных на этапах передачи синтаксической информации;
  \item обеспечить оптимизацию взаимодействия парсера и онтологии.
\end{list}

Достижение вышеупомянутых целей позволит увеличить эффективность работы специалиста предметной области, избавив его необходимости формировать наполнение словарей грамматик, упростит дальнейшее развитие парсера.

\section*{A3 Характеристика объекта автоматизации}
\addcontentsline{appatoc}{section}{A3 Характеристика объекта автоматизации}
\subsection*{A3.1 Краткие сведения об объекте автоматизации}
\addcontentsline{appatoc}{subsection}{A3.1 Краткие сведения об объекте автоматизации}
Объектом автоматизации является деятельность специалиста по извлечению знаний на этапе пополнения им словарей грамматик синтаксического пасрера.

\subsection*{A3.2 Сведения об условиях эксплуатации и окужающей среде}
\addcontentsline{appatoc}{subsection}{A3.2 Сведения об условиях эксплуатации и окужающей среде}
Оборудование, используемое для функционирования модуля находится в нормальных для текущего географического
региона условиях окружающей среды.

График функционирования модуля совпадает с рабочим графиком специалиста предметной области, должна быть предусмотрена возможность круглосуточного функционирования в течение длительного срока.

\section*{A4 Требования к системе}
\addcontentsline{appatoc}{section}{A4 Требования к системе}
\subsection*{Требования к системе в целом}
\addcontentsline{appatoc}{subsection}{A4.1 Требования к системе в целом}
\subsubsection*{A4.1.1 Требования к структуре и функционированию системы}
\addcontentsline{appatoc}{subsubsection}{A4.1.1 Требования к структуре и функционированию системы}

Модуль должен выполнять следующие функции:
\begin{list}{\labelitemi}{\leftmargin=1.5cm}
  \item формировать правила грамматики связей по запросу парсера на основе информации из внешней онтологии;
  \item пополнять словарь грамматики связей, используемый парсером;
  \item при необходимости передавать онтологии правила, используемые парсером, но отсутствующие в онтологии;
  \item кэшировать полученные данные с целью оптимизации функционирования.
\end{list}

Модуль также должен удовлетворять следующим требованиям:
\begin{list}{\labelitemi}{\leftmargin=1.5cm}
  \item быть простым в эксплуатации --- не требовать дополнительного обучения администраторов и пользователей;
  \item быть надёжным --- должным образом функционировать в течение всего требуемого времени;
  \item быть заменяемым --- предоставлять возможность отключения модуля и возврата к стандартной схеме функционирования без модификации исходной системы;
  \item быть конфигурируемым --- предоставлять возможность специалисту установить собственные настройки работы модуля без вмешательства разработчика.
\end{list}

Модуль должен состоять из следующих подсистем:
\begin{list}{\labelitemi}{\leftmargin=1.5cm}
  \item контроллер (ядро модуля);
  \item интерфейс к парсеру;
  \item интерфейс к онтологии;
  \item подсистема кэширования;
  \item подсистема генерации запросов;
  \item подсистема трансляции ответов.
\end{list}

\paragraph{Контроллер} модуля должен обеспечивать связь и согласованное взаимодействие компонентов между собой.

Функции контроллера модуля:
\begin{list}{\labelitemi}{\leftmargin=1.5cm}
  \item обеспечивать корректную инициализацию модуля;
  \item осуществлять чтение и установку конфигурации модуля;
  \item осуществлять синхронизацию взаимодействия остальных подсистем между собой.
\end{list}

Требования к контроллеру модуля:
\begin{list}{\labelitemi}{\leftmargin=1.5cm}
  \item должен быть скомпонован в виде разделяемой библиотеки кода;
  \item должен обладать высокой производительностью;
  \item должен обеспечивать приемлемый уровень отказоустойчивости и обработку исключений.
\end{list}

\paragraph{Интерфейс к парсеру} должен организовывать обмен данными между парсером и контроллером модуля.

Функции интерфейса к парсеру:
\begin{list}{\labelitemi}{\leftmargin=1.5cm}
  \item предоставлять парсеру функции для получения правил посредством API парсера;
  \item осуществлять преобразование данных из формата парсера во внутренний формат модуля.
\end{list}

Требования к интерфейсу к парсеру:
\begin{list}{\labelitemi}{\leftmargin=1.5cm}
  \item должен поддерживать текущее API Link Grammar парсера;
  \item должен быть гибким, с возможностью адаптации к другим API. 
\end{list}

\paragraph{Интерфейс к онтологии} должен организовывать обмен данными между онтологией и контроллером модуля.

Функции интерфейса к онтологии:
\begin{list}{\labelitemi}{\leftmargin=1.5cm}
  \item подключаться к внешним онтологиям посредством протокола HTTP;
  \item передавать онтологии SPARQL-запросы;
  \item получать от онтологии RDF-ответы и передавать их контроллеру модуля.
\end{list}

Требования к интерфейсу к онтологии:
\begin{list}{\labelitemi}{\leftmargin=1.5cm}
  \item должен предоставлять возожность конфигурации;
  \item должен поддерживать протоколы передачи данных HTTP и HTTPS.
\end{list}

\paragraph{Подсистема кэширования} должна хранить полученные от онтологии данных для более быстрого повторного доступа к ним.

Функции кэша:
\begin{list}{\labelitemi}{\leftmargin=1.5cm}
  \item хранить полученные от транслятора правил правила;
  \item хранить элементы запросов;
  \item хранить вспомогательную информацию.
\end{list}

Требования к кэшу:
\begin{list}{\labelitemi}{\leftmargin=1.5cm}
  \item должен быть реализован на основе KVS;
  \item должен осуществлять журнализацию для организации перзистентного хранения данных;
  \item должен быть производительным.
\end{list}

\paragraph{Подсистема генерации запросов} должна генерировать SPARQL-запросы на по требованию парсера.

Функции подсистемы генерации запросов:
\begin{list}{\labelitemi}{\leftmargin=1.5cm}
  \item формировать SPARQL-запросы на основе макетов из кэша и слов из парсера;
  \item валидировать запросы перед их отправкой.
\end{list}

Требования к подсистеме генерации запросов:
\begin{list}{\labelitemi}{\leftmargin=1.5cm}
  \item должна реализовывать стандарты SPARQL, регламентированные консорциумом Semantic Web;
  \item должна кэшировать информацию для повышения производительности;
  \item должна обеспечивать гарантированную генерацию запроса на основе любого слова.
\end{list}

\paragraph{Подсистема трансляции ответов} должна преобразовывать полученный от онтологии RDF в строковые правила грамматики связей.

Функции подсистемы трансляции ответов:
\begin{list}{\labelitemi}{\leftmargin=1.5cm}
  \item извлекать из RDF сущности и атрибуты, релевантные запросам парсера;
  \item комбинировать извлечённые данные в виде правил грамматики связей.
\end{list}

Требования к подсистеме трансляции ответов:
\begin{list}{\labelitemi}{\leftmargin=1.5cm}
  \item должна полностью поддерживать стандарт RDF, принятый консорциумом Semantic Web;
  \item должна обеспечивать возврат правила (в том числе пустого) при любых входных данных.
\end{list}

\subsubsection*{A4.1.2 Требования к способам и средствам связи для информационного обмена между компонентами системами}
\addcontentsline{appatoc}{subsubsection}{A4.1.2 Требования к способам и средствам связи для информационного обмена между компонентами системы}
Обмен данными внутри модуля осуществляется с помощью потоков ввода-вывода используемой операционной системы, с помощью разделяемых ресурсов и с помощью прямой передачи указателей на хранимые в памяти данные. Взаимодействие со внешними сущностями осуществляется посредством протокола HTTP и сети Интернет.

\subsubsection*{A4.1.3 Требования к совместимости}
\addcontentsline{appatoc}{subsubsection}{A4.1.3 Требования к совместимости}
Модуль должен быть совместим с операционными системами семейства GNU/Linux и Microsoft Windows. Работа на иных Unix-подобных системах возможна, но не требуется.

\subsubsection*{A4.1.4 Требования к режимам функционирования системы}
\addcontentsline{appatoc}{subsubsection}{A4.1.4 Требования к режимам функционирования системы}

\paragraph{Штатный режим}: в этом режиме модуль выполняет функции основного источника информации для парсера, используя данные онтологии.

\paragraph{Аварийный режим}: в этом режиме модуль приостанавливает свою работу, передавая парсеру информацию из его словарей.

\subsubsection*{A4.1.5 Требования по диагностированию системы}
\addcontentsline{appatoc}{subsubsection}{A4.1.5 Требования по диагностированию системы}
Модуль должен осуществлять полное журналирование процесса своей работы в текстовые файлы, коды и описания исключений, входящией данные, вызвавшие их возникновение.

\subsubsection*{A4.1.6 Требования к показателям назначения}
\addcontentsline{appatoc}{subsubsection}{A4.1.6 Требования к показателям назначения}

Модуль должен осуществлять единичный цикл обработки запроса/трансляции ответа не более, чем за одну секунду, без учёта сетевых задержек и скорости ответа сервера онтологии.

Модуль должен инициализироваться одновременно с парсером и включаться в работу при соответствующей конфигурации.

Модуль должен обеспечивать отказоустойчивость и журналирование исключительных ситуаций. Исключительные ситуации могут прерывать единичную итерацию, но не могут вызывать сбоев в работе модуля либо парсера.

Модуль должен автоматически переводить парсер на работу со словарём в случае недоступности онтологии.

Модуль должен обеспечивать распараллеливание работы, число потоков обработки должно соответствовать числу ядер используемого процессора.

\subsection*{A4.2 Требования к функциям (задачам), выполняемым системой}
\addcontentsline{appatoc}{subsection}{A4.2 Требования к функциям (задачам), выполняемым системой}

\subsubsection*{A4.2.1 Подсистема контроллера модуля}
\addcontentsline{appatoc}{subsubsection}{A4.2.1 Подсистема контроллера модуля}

Должна обеспечивать взаимодействие компонентов модуля между собой, устойчивую
работу модуля в целом, связь с онтологией и парсером.

Контроллер выполняет роль точки входа библиотеки модуля. В штатном режиме он распределяет данные между компонентами модуля, в аварийном работает прозрачно между парсером и его словарями.

Контроллер выполняет следующие функции:
\begin{list}{\labelitemi}{\leftmargin=1.5cm}
  \item запускает потоки компонентов приложения;
  \item осуществляет обработку исключительных ситуаций;
  \item журналирует процесс работы модуля.
\end{list}

Контроллер модуля должен быть реализован в первую очередь в виде разделяемой библиотеки кода. Планируется его реализация на основе Haskell Platform и Haskell FFI.

Срок реализации контроллера: с 01.03.2013 по 08.03.2013.

\subsubsection*{A4.2.2 Подсистема взаимодействия с парсером}
\addcontentsline{appatoc}{subsubsection}{A4.2.2 Подсистема взаимодействия с парсером}

Должна передавать запросы парсера в подсистему генерации запросов.

Выполняет следующие функции:
\begin{list}{\labelitemi}{\leftmargin=1.5cm}
  \item принимает вызов функции от парсера;
  \item генерирует уникальный идентификатор итерации;
  \item преобразует формат данных из используемого парсером в формат, используемый модулем.
\end{list}

Должна быть реализована в срок с 01.03.2013 по 08.03.2013, одновременно с реализацией контроллера модуля.

\subsubsection*{A4.2.3 Подсистема генерации запросов}
\addcontentsline{appatoc}{subsubsection}{A4.2.3 Подсистема контроллера модуля}

Должна обеспечивать формирования SPARQL-запросов на основе слов, полученных от парсера.

Генератор запросов осуществляет следующие функции:
\begin{list}{\labelitemi}{\leftmargin=1.5cm}
  \item принимает вызов функции от парсера;
  \item запрашивает из кэша макеты запросов;
  \item подставляет запращиваемое слово в запрос;
  \item валидирует конечный запрос на соответствие стандарту.
\end{list}

Реализация генератора запросов должна производиться непосредственно после реализации контроллера модуля и до начала реализации модуля трансляции ответов и интерфейса к онтологии.

Срок реализации: с 08.03.2013 по 29.03.2013.

\subsubsection*{A4.2.4 Подсистема взаимодействия с онтологией}
\addcontentsline{appatoc}{subsubsection}{A4.2.4 Подсистема взаимодействия с онтологией}

Должна отправлять онтологии запросы, полученные от генератора, принимать ответы и передавать их в транслятор ответов.

Интерфейс к онтологии выполняет следующие функции:
\begin{list}{\labelitemi}{\leftmargin=1.5cm}
  \item принимает запрос от генератора;
  \item открывает соединение с онтологией;
  \item оборачивает запрос в форму пакета используемого сетевого протокола;
  \item передаёт запрос онтологии и ожидает ответа;
  \item принимает ответ и преобразует его во внутренний формат;
  \item передаёт ответ онтологии в транслятор ответов;
  \item закрывает соединение с онтологией.
\end{list}

Подсистема должна быть реализована после подсистемы генерации запросов и до начала реализации компонента трансляции ответов.

Срок реализации: с 29.03.2013 по 05.04.2013.

\subsubsection*{A4.2.5 Подсистема трансляции ответов}
\addcontentsline{appatoc}{subsubsection}{A4.2.5 Подсистема трансляции ответов}

Должна преобразовывать полученный от онтологии RDF в строковые правила грамматики связей.

Транслятор ответов выполняет следующие функции:
\begin{list}{\labelitemi}{\leftmargin=1.5cm}
  \item получает RDF от компонента взаимодействия с онтологией;
  \item анализирует полученный RDF, извлекая из него сущности, соответствующие шаблонам;
  \item выделяет связи и сортирует их по обеим сторонам слова;
  \item формирует итоговое правило и отправляет его контроллеру.
\end{list}

Подсистема должна быть реализована после подсистемы интерфейса к онтологии одновременно с подсистемой кэширования.

Срок реализации: с 05.04.2013 по 26.04.2013.

\subsubsection*{A4.2.6 Подсистема кэширования}
\addcontentsline{appatoc}{subsubsection}{A4.2.6 Подсистема кэширования}

Осуществляет кэширование полученных в итерации правил с целью оптимизации работы модуля.

\begin{list}{\labelitemi}{\leftmargin=1.5cm}
  \item принимает кортеж значений (идентификатор итерации, слово, правило);
  \item запоминает полученные значения;
  \item ищет по запросу соответствие правила слову.
\end{list}

Подсистема должна быть реализована в срок с 05.04.2013 по 09.04.2013 после реализации всех остальных компонент модуля.

\subsection*{A4.3 Требования к видам обеспечения}
\addcontentsline{appatoc}{subsection}{A4.3 Требования к видам обеспечения}

\subsubsection*{A4.3.1 Требования к математическому обеспечению}
\addcontentsline{appatoc}{subsubsection}{A4.3.1 Требования к математическому обеспечению}

Математические методы и алгоритмы, используемые для шифрования и дешифрования
данных, а также программное обеспечение, реализующее их, должны быть
сертифицированы уполномоченными организациями для использования в
государственных органах Российской Федерации.

\subsubsection*{A4.3.2 Требования к информационному обеспечению}
\addcontentsline{appatoc}{subsubsection}{A4.3.2 Требования к информационному обеспечению}

Хранение данных в модуле должно быть организовано на основе современных
документ"=ориентированных БД либо хранилищ типа <<ключ-значение>>.

Информация должна журналироваться, после перезапуска СУБД должно быть возможным её восстановление к состоянию до перезапуска. 

\subsubsection*{A4.3.3 Требования к применению языков высокого уровня}
\addcontentsline{appatoc}{subsubsection}{A4.3.3 Требования к применению языков высокого уровня}

Используемые при разработке языки высокого уровня должны обеспечивать решение
всех задач по реализации функций системы.

Допускается использование стандартных языков высокого уровня, отвечающих
требованиям реализации задач предметной области.

Используемые языки должны предоставлять возможность исполнения результирующего
кода на различных программных и аппаратных платформах.

\subsubsection*{A4.3.4 Требования к программному обеспечению}
\addcontentsline{appatoc}{subsubsection}{A4.3.4 Требования к программному обеспечению}

При выборе ПО предпочтение должно отдаваться архитектурным решениям и
программным продуктам, уже доказавшим свою пригодность при решении подобных
задач. Базовое ПО должно поддерживать и использовать стандартные сетевые
протоколы передачи данных. Базовой программной платформой должна являться
операционная система GNU/Linux, должна поддерживаться также MS Windows XP/Vista/7/8.