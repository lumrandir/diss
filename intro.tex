\section*{\centering ВВЕДЕНИЕ}
\addcontentsline{toc}{section}{ВВЕДЕНИЕ}
Компьютерная лингвистика --- быстро развивающаяся область на стыке лингвистики, математики
и информатики. Достижения в этой области позволяют осуществлять обработку информации, представленной
в самой распространённой форме --- текстов на естественном языке. Такая обработка может осуществляться
автоматизированно, при минимальной поддержке со стороны человека, либо полностью автоматически, используя в качестве
источника данных обширнейшую коллекцию текстов в Интернете.

Работы по NLP ведутся с середины XX в. \cite{wiki_nlp}, и к настоящему времени разработана подробная методология
обработки касательно всех аспектов естественно-языковых данных (морфология, синтаксис, семантика и т.д.), реализовано
множество систем автоматической обработки естественного языка, в том числе парсеров. Существуют формализмы и языки программирования, 
изначально разрабатывавшиеся для решения проблемы обработки символьной информации (LISP, Рефал).

Большинство проблемных ситуаций в автоматическом парсинге естественных языков вызвано несовершенством формальных грамматик, представляющих анализируемый язык внутри алгоритма обработки. Естественные языки плохо поддаются формализации, формальные грамматики неоднозначны и могут порождать на одном и том же высказывании множество различных результатов анализа, некоторые из которых будут полностью лишены смысла для человека.

В свою очередь, совершенствованию и уточнению существующих грамматик мешает отсутствие централизованных источников синтакических знаний, в результате чего каждый новый парсер должен иметь собственную базу правил используемой в алгоритме формальной грамматики. Помимо трудоёмкости формирования такой базы, значительно возрастает риск ошибки при наполнении базы, снижается полнота представленной в ней информации.

Концепция Semantic Web \cite{wiki_semantic_web}, с другой стороны, предлагает фреймворк, в рамках которого различные системы и приложения получают доступ к централизованным источникам знаний (например, в виде онтологий). Такой централизованный источник может обладать самой полной и корректной базой знаний об определённой предметной области (например, о синтаксисе конкретного естественного языка) и формироваться специалистами. Таким образом, использование внешних баз знаний при парсинге может существенно снизить неоднозначность, упростить разработку новых алгоритмов, сместив центр внимания на совершенствование используемого алгоритма, а не грамматики.

С этой точки зрения, наибольшей актуальностью подобная технология обладает именно в приложении к парсерам русского языка в связи с большей сложностью его формализации по сравнению с английским, для которого разработано и применяется множество различных методов парсинга.