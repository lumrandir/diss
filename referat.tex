\section*{\centering РЕФЕРАТ}

Пояснительная записка \total{page} страниц, \total{figure} рисунков, \total{table}
таблиц, \total{citnum} источников.

\emph{Актуальность работы}. С учётом тенденций к консолидации лингвистической 
информации и распространению технологий semantic web, наблюдается неспособность
имеющихся синтаксических парсеров русского языка к работе в меняющемся окружении. 
Следовательно, необходимо выделить наиболее эффективное решение из имеющихся
парсеров и модифицировать его, приспособив к работе с централизованными
хранилищами лингвистической информации.

\emph{Целью исследования} является развитие теоретического базиса технологии
взаимодействия синтаксических парсеров русского языка с централизованными
хранилищами лингвистической информации и разработка демонстрационного программного
продукта, эксплуатирующего данную технологию.

Для достижения поставленной цели необходимо решить следующие \emph{задачи}:
\begin{list}{\labelitemi}{\leftmargin=1.5cm}
  \item найти аналогичные синтаксические парсеры русского языка;
  \item выделить критерии сравнительного анализа аналогов;
  \item среди найденных парсеров"=аналогов определить парсер"=прототип,
  наиболее полно удовлетворяющий выделенным критериям;
  \item исследовать существующие технологии взаимодействия с онтологиями \\
  semantic  web;
  \item разработать пакет моделей взаимодействия существующей системы с
  централизованными хранилищами лингвистической информации;
  \item разработать техническое задание на разработку модуля взаимодействия выбранного
  в качестве прототипа парсера с онтологиями semantic web;
  \item произвести проектирование описанного в техническом задании модуля;
  \item выполнить инженерную реализацию спроектированного модуля.
\end{list}

\emph{Объектом исследования} является синтаксический парсер русского языка.

\emph{Предметом исследования} являются технологии взаимодействия программных
средств с онтологиями semantic web.

\emph{Методы исследования}. Для разработки системы использовались методы
системотехники, проектирования информационных систем и технологии разработки
программного обеспечения.

\emph{Научная новизна работы}. Получен пакет моделей взаимодействия синтаксического
парсера русского языка с онтологиями Semantic Web, описывающих модуль, заменяющий собой традиционные текстовые словари прототипа.
